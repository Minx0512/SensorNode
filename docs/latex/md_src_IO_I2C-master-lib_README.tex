\subsection*{Table of contents}


\begin{DoxyEnumerate}
\item What is this library?
\item Contents of this library
\item Supported devices
\item Functions in this library 


\end{DoxyEnumerate}

\subsubsection*{1. What is this library?}

This library is an I2C master library which uses the T\+WI peripheral inside the A\+VR microcontroller to establish connections using the I2C protocol 



\subsubsection*{2. Contents of this library}

The library contains the following files\+:


\begin{DoxyItemize}
\item main.\+c
\item I2\+C\+\_\+master.\+c
\item I2\+C\+\_\+master.\+h
\end{DoxyItemize}

\subparagraph*{main.\+c}

This is a piece of example code which uses this library to read out an H\+M\+C5338L I2C magnetometer.

\subparagraph*{I2\+C\+\_\+master.\+c}

This file contains all the function declarations to setup and work with the T\+WI hardware peripheral inside the A\+VR.

{\itshape Make sure you add this source file to your Makefile!}

\subparagraph*{I2\+C\+\_\+master.\+h}

This file contains the function prototypes and the definition of the Read / Write bit (0 = read, 1 = write)

{\itshape This file has to be included in your source file} 



\subsubsection*{3. Supported devices}

Though I have only tested this library on an A\+Tmega328P it should be running on all major A\+Tmega A\+V\+Rs like\+:


\begin{DoxyItemize}
\item A\+Tmega644
\item A\+Tmega32
\item A\+Tmega16
\item A\+Tmega328/P
\item A\+Tmega168/P
\item A\+Tmega88/P
\item A\+Tmega44/P
\item A\+Tmega8
\end{DoxyItemize}

If your device is not supported you can probably adapt this library easily to your needs by having a look at the your device\textquotesingle{}s data sheet and changing the register names appropriately

The A\+Ttiny range of microcontrolles is not supported as they only provide a U\+SI peripheral which is completely different from the T\+WI peripheral used on the other controllers 



\subsubsection*{4. Functions in this library}


\begin{DoxyItemize}
\item void i2c\+\_\+init(void)
\item uint8\+\_\+t i2c\+\_\+start(uint8\+\_\+t address)
\item uint8\+\_\+t i2c\+\_\+write(uint8\+\_\+t data)
\item uint8\+\_\+t i2c\+\_\+read\+\_\+ack(void)
\item uint8\+\_\+t i2c\+\_\+read\+\_\+nack(void)
\item uint8\+\_\+t i2c\+\_\+transmit(uint8\+\_\+t address, uint8\+\_\+t$\ast$ data, uint16\+\_\+t length)
\item uint8\+\_\+t i2c\+\_\+receive(uint8\+\_\+t address, uint8\+\_\+t$\ast$ data, uint16\+\_\+t length)
\item void i2c\+\_\+stop(void)
\end{DoxyItemize}

\subparagraph*{void I2\+C\+\_\+init(void)}

This function needs to be called only once to set up the correct S\+CL frequency for the bus

\subparagraph*{uint8\+\_\+t I2\+C\+\_\+start(uint8\+\_\+t address)}

This function needs to be called any time a connection to a new slave device should be established. The function returns 1 if an error has occurred, otherwise it returns 0.

The syntax to start a operation write to a device is either\+: {\ttfamily I2\+C\+\_\+start(S\+L\+A\+V\+E\+\_\+\+A\+D\+D\+R\+E\+S\+S+\+I2\+C\+\_\+\+W\+R\+I\+TE);} or {\ttfamily I2\+C\+\_\+start(\+S\+L\+A\+V\+E\+\_\+\+W\+R\+I\+T\+E\+\_\+\+A\+D\+D\+R\+E\+S\+S);}

The syntax to start a read operation from a device is either\+: {\ttfamily I2\+C\+\_\+start(S\+L\+A\+V\+E\+\_\+\+A\+D\+D\+R\+E\+S\+S+\+I2\+C\+\_\+\+R\+E\+AD);} or {\ttfamily I2\+C\+\_\+start(\+S\+L\+A\+V\+E\+\_\+\+R\+E\+A\+D\+\_\+\+A\+D\+D\+R\+E\+S\+S);}

\subparagraph*{uint8\+\_\+t I2\+C\+\_\+write(uint8\+\_\+t data)}

This function is used to write data to the currently active device. The only parameter this function takes is the 8 bit unsigned integer to be sent. The function returns 1 if an error has occurred, otherwise it returns 0.

\subparagraph*{uint8\+\_\+t I2\+C\+\_\+read\+\_\+ack(void)}

This function is used to read one byte from a device and request another byte of data after the transmission is complete by sending the acknowledge bit. This function returns the received byte.

\subparagraph*{uint8\+\_\+t I2\+C\+\_\+read\+\_\+nack(void)}

This function is used to read one byte from a device an then not requesting another byte and therefore stopping the current transmission. This function returns the received byte.

\subparagraph*{uint8\+\_\+t i2c\+\_\+transmit(uint8\+\_\+t address, uint8\+\_\+t$\ast$ data, uint16\+\_\+t length);}

This function is used to transmit \mbox{[}length\mbox{]} number of bytes to an I2C device with the given I2C address from \mbox{[}data\mbox{]}. The \mbox{[}address\mbox{]} passed to this function is the 7-\/bit slave address, left shifted by one bit (i.\+e. 7-\/bit slave address is {\ttfamily 0x2F} -\/$>$ {\ttfamily (0x2F)$<$$<$1} = {\ttfamily 0x5E})

\subparagraph*{uint8\+\_\+t i2c\+\_\+receive(uint8\+\_\+t address, uint8\+\_\+t$\ast$ data, uint16\+\_\+t length);}

This function is used to read \mbox{[}length\mbox{]} number of bytes from the I2C device with the given I2C address into the \mbox{[}data\mbox{]}. The \mbox{[}address\mbox{]} passed to this function is the 7-\/bit slave address, left shifted by one bit (i.\+e. 7-\/bit slave address is {\ttfamily 0x2F} -\/$>$ {\ttfamily (0x2F)$<$$<$1} = {\ttfamily 0x5E})

\subparagraph*{void I2\+C\+\_\+stop(void)}

This function disables the T\+WI peripheral completely and therefore disconnects the device from the bus. 